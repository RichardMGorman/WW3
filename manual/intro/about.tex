\vssub
\subsection{~About this manual}
\vssub

This document describes the governing equations (\textbf{Chapter ~\ref{chapt:eq}}), numerical approaches (\textbf{Chapter~\ref{chapt:num}}),
model structure and data flow (\textbf{Chapter~\ref{chapt:run}}), installing, compiling and running (\textbf{Chapter~\ref{chapt:impl}}) of \ws. Further details on the general code 
structure and implementation of different aspects are given in \textbf{Chapter~\ref{chapt:sys}}. A user wishing to install the model 
may thus jump directly to \textbf{Chapter~\ref{chapt:impl}}, and then successively
modify input files in example runs (Chapter~\ref{chapt:run}). However this will not replace a thorough knowledge of \ws\ that can be obtained by
following Chapters~\ref{chapt:eq} through \ref{chapt:impl}.



This is the user manual and system documentation of version \WWver\ of the
third-generation wind-wave modeling framework \ww. While code management of this system is undertaken by the National Center for Environmental Prediction (NCEP) the model development relies on a community of developers (see below). It is based on WAVEWATCH~I and WAVEWATCH~II as developed at
Delft University of Technology, and NASA Goddard Space Flight Center,
respectively. \ws\ differs from its predecessors in all major aspects; i.e.,
governing equations, program structure, numerical and physical approaches.



The format of a combined user manual and
system documentation has been chosen to give users the necessary background
to include new physical and numerical approaches in the framework according to
their own specifications.  This approach became more important as \ws\
developed into a wave modeling framework. By design, a user can apply his or
her numerical and/or physical approaches, and thus develop a new wave model based
on the \ws\ framework. In such an approach, optimization, parallelization,
nesting, input and output service programs from the framework can be easily
shared between actual models.  Whereas this document is intended to be
complete and self-contained, this is not the case for all elements in the
system documentation. For additional system details, reference is made to the
source code, which is fully documented. Note that a best practices guide for
code development for \ws\ is now available \citep{tol:MMAB10a, tol:MMAB14b}.

\vspace{\baselineskip} 
\noindent 
The present model version (\WWver) is the new public version based on the
last official model release (version 5.16). Since the latter release the
following modifications have been made:

\begin{list}{$\bullet$}{\rightmargin 5mm \parsep 0mm \itemsep 0mm}

\item Preparing for next model version, adding optional instrumentation to code
      for profiling of memory use (model version 6.00).

\item Separates Stokes drift spectrum calculation (US3D) from OUTG and provides 
      new option to output surface Stokes drift partitioned into run-time defined 
      frequencies (USSP) (model version 6.01).  

\item Adds a new module for ESMF interface (model version 6.02). 

\item Adds a capability to update restart file's total energy based on independent 
      significant wave height analysis (model version 6.03).

\item Adds domain decomposition for unstructured implicit schemes using PDLIB
      (Parallel Domain Decomposition Library) and ParMetis (model version 6.04).

\item Updates the namelist options for the following programs: ww3\_ounf, ww3\_ounp, 
      ww3\_trnc, ww3\_bounc, and ww3\_shel (model version 6.05).

\end{list}

\vspace{\baselineskip} \noindent 
Up to date information on this model can be found (including bugs and bug
fixes) on the \ws\ web page, 
\begin{center}
\url{http://polar.ncep.noaa.gov/waves/wavewatch/}
\end{center}
and comments, questions and suggestions should be
directed to the code manager, Jessica Meixner (jessica.meixner@noaa.gov), or the general \ws\ users mailing group list

\begin{center}
ncep.list.wwatch3.users@lstsrv.ncep.noaa.gov
\end{center}

\noindent
NCEP will redirect questions regarding contributions from outside NCEP to the
respective authors of the codes. You may subscribe to the \ws\ users 
mailing list at the following web page:
\begin{center}
\footnotesize
\url{https://www.lstsrv.ncep.noaa.gov/mailman/listinfo/ncep.list.wwatch3.users}
\end{center} 

