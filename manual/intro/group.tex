\vssub
\subsection{~The WAVEWATCH III\textsuperscript{\textregistered} Development Group (WW3DG) \label{sec:WW3DG}}
\vssub

The development of \ww  relies on the efforts of a team of developers that have worked tirelessly to make this an effective community tool. With the expansion of physical and numerical parameterizations available, the list of contributors to this model keeps growing. The development group consists of a core group of developers that are involved in overall code development, debugging and optimization as well as a larger group that has either made or continues to make contributions to physics packages and numerics. The following is a list of contributors (both past and present) of this development group (in alphabetic order):

\begin{list}{\ldots}{ }

\item [Mickael Accensi] (Ifremer, France) \\
  NetCDF for input and output (ww3\_prnc, ww3\_ounf, ww3\_ounp), namelist input files for ww3\_multi, and general code development support.

\item [Jose-Henrique Alves] (SRG at NOAA/NCEP/EMC, USA) \\
  Support of code development at NCEP, shallow water physics packages, development of space-time wave-height extremes approach.

\item [Fabrice Ardhuin] (CNRS, France, previously at SHOM then Ifremer) \\
  Various physics packages (ST3, ST4, BS1, BT4, IG1, REF1, IS2...), interface with unstructured grid schemes, tidal analysis, and some I/O aspects (estimation of fluxes, adaptation of NetCDF). 

\item [Alexander Babanin] (University of Melbourne, Australia)\\
  ST6 project leader, source functions (wind input, whitecapping dissipation, swell dissipation, negative input, physical constraints)

\item [Francesco Barbariol] (ISMAR-CNR, Italy) \\
  Development of a space-time wave-height extremes approach.

\item [Alvise Benetazzo] (ISMAR-CNR, Italy) \\
  Development of a space-time wave-height extremes approach.

\item [Anne-Claire Bennis] (University of Caen, France, previously at SHOM, France) \\
  Coupling with 3D flow model using PALM.

\item [Jean Bidlot] (ECMWF, UK) \\
  Updates to physics package ST3.

\item [Nico Booij] (Delft University of Technology, The Netherlands, retired) \\
  Original design of source code pre-processor ({\code w3adc}), basic method
  of documentation and other programming habits. Spatially varying wavenumber
  grid.

\item [Guillaume Boutin] (Ifremer, France) \\
  Contribution to IS2 and IC2. 

\item [Tim Campbell] (Naval Research Laboratory, USA)\\ 
  Search and regrid utilities, irregular grids, regression testing shell script, and overall code development support.

\item [Dmitry V. Chalikov] (Formerly UCAR at NOAA/NCEP/EMC) \\ Co-author of the
  \cite{tol:JPO96} input and dissipation parameterizations and source code.

\item [Arun Chawla](NOAA/NCEP/EMC, USA) \\
  Support of code development at NCEP, GRIB packing, automated grid generation
  software \citep{tol:MMAB07a, tol:OMOD08a}.

\item [Sukun Cheng] (while at Clarkson University, USA) \\
  Original author of the code that was ported into WW3 (for model version 5) as the improved ``IC3'' parameterization for effect of sea ice on waves.

\item [Clarence Collins] (while an NRL/ASEE post-doc, USA) \\
  Origination of IC4 (sea ice source function).

\item [Jean-Fran{\c c}ois Filipot] (France Energy Marine, formerly at SHOM then Ifremer, France).\\
  Unification of whitecapping and breaking in ST4. 

\item [Stylianos -Stelios- Flampouris] (IMSG at NOAA/NCEP/EMC, USA) \\
      Wave Data Assimilation, communication between DA systems and the model (ww3\_uprstr) 

\item [Mike Foreman]  (IOS, Canada) \\
  Versatile tidal analysis package. 

\item [Isaac Ginis] (University of Rhode Island, USA) \\
  Development of source code for sea-state dependent wind stress calculations (FLD1, FLD2).

\item [Tetsu Hara] (University of Rhode Island, USA) \\
  Development of source code for sea-state dependent wind stress calculations (FLD1, FLD2).

\item [Peter Janssen] (ECMWF, United Kingdom) \\
  Original version of WAM-Cycle 4 package (ST3), canonical transform for the second order wave spectrum.

\item [Fabien Leckler] (Ifremer, France) \\
  Breaking parameters from source terms and contributions to ST4.

\item [Jian-Guo Li] (UK MetOffice, United Kingdom) \\
  SMC grid, second order UNO schemes and rotated grids.

\item [Kevin Lind]  (DoD PETTT, USA)\\ 
  Improvements to performance of some multi-grid functions.

\item [Jessica Meixner] (IMSG at NOAA/NCEP/EMC, USA) \\
  Coupled modeling development, tripole grids, general code development support  and code manager for \ws.
 
\item [Mark Orzech]  (Naval Research Laboratory, USA)\\ 
  Source terms for effects of mud (BT8, BT9).

\item [Roberto Padilla--Hern\'andez]  (IMSG at NOAA/NCEP/EMC, USA)\\ 
  Support of code development at NCEP, editing.

\item [William Perrie] (Bedford Institute of Oceanography, Canada)\\
 Two-Scale Approximations for non-linear interactions (NL4).

\item [Arshad Rawat] (MIO, Mauritius and Ifremer, France) \\
  Contribution to second order spectrum and free infragravity wave sources (IG1).    

\item [Brandon Reichl] (NOAA/GFDL and Princeton University; Formerly at University of Rhode Island, USA) \\
  Development and coding of source code for sea-state dependent wind stress calculations (FLD1, FLD2).

\item [W. Erick Rogers]  (Naval Research Laboratory, USA)\\ 
  Irregular grids, source terms for effects of sea ice (e.g. in IC1, IC2, IC3, IC4, IC5) and mud (BT8, BT9), adaptation/interfacing of conservative remapping software, tripole grid, regression tests, and overall code development support.

\item [Aron Roland] (T. U. Darmstadt, Germany) \\
  Advection on unstructured (triangle-based) grids and meshing tools.

\item [Caroline Sevigny] (UQAR, Canada) \\
  Contribution to ice scattering including ice break-up.

\item [Hayley Shen] (Clarkson Univ.) \\
  Supervised contributions by Zhao and Cheng on the ``IC3'' parameterization for effect of sea ice on waves.

\item [Mathieu Dutour Sikiric] (IRB, Croatia) \\
  Multi-grid computations with unstructured (triangle-based) grids.

\item [Mark Szyszka] (RPS Group, Australia) \\
  Identifying several bugs in the code development process and providing fixes for Openmp issues.

\item [Hendrik L. Tolman] (DOC/NOAA/NWS/OSTI, USA). \\
  General code architecture, original \wt-I, II and III models. Ongoing model
  development.

\item [Bash Toulany] (Bedford Institute of Oceanography, Canada)\\
 Two-Scale Approximations for non-linear interactions (NL4).

\item [Barbara Tracy] (US Army Corps of Engineers, ERDC-CHL, USA, retired) \\
  Spectral partitioning.

\item [Gerbrant Ph. van Vledder] (Delft University of Technology, NL) \\
  Webb-Resio-Tracy exact nonlinear interaction routines, as well as some of
  the original service routines.

\item [Andr\'e van der Westhuysen](IMSG at NOAA/NCEP/EMC, USA) \\
  Support of code development at NCEP, wave system tracking, addition of triad interactions.

\item [Ian Young] (University of Melbourne, Australia)
  ST6 source functions (wind input, whitecapping dissipation).

\item [Xin Zhao] (while at Clarkson University, USA) \\
  Original author of the code that was ported into WW3 (model version 4) as the ``IC3'' parameterization for effect of sea ice on waves.

\item [Stefan Zieger] (Bureau of Meteorology, Australia) \\
  ST6 source term package, code and testing.
\end{list}
