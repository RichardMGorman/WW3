        `\vssub
\subsubsection{~Quadtree adaptive grids} \label{sub:num_space_qa}
\conthead{\ws\ (NIWA)}{R. Gorman}

\noindent 
Wave simulations are often required to incorporate a wide range of spatial 
scales, in order to capture both long-range swell propagation and finer scale 
interactions either near the coast or within rapidly-varying small-scale 
weather systems.
Computational efficiency can be greatly aided by allowing a model's spatial 
resolution to vary over the domain, with finer resolution where it makes most 
impact on the accuracy of the solution.
In the case where the need for locally higher resolution is static,
for example due to highly variable nearshore topography, this can be addressed 
by nested regular grids, or by various forms of unstructured meshes.
Alternatively, high resolution may be needed to capture the influence of
rapidly varying wind fields in the vicinity of a tropical cyclone.
In this case, we can use adaptive mesh refinement to maintain an efficient 
distribution of spatial resolution.
This has previously been applied to spectral wave modelling 
\citep{art:PGRT10xx} 
by using the Gerris flow solver \citep{art:PopinetXX} to compute the spatial 
advection terms of the action balance action on a quadtree adaptive mesh, 
coupled to \ws\ V2.22 subroutines to handle a limited set of source terms.

Quadtree adaptive mesh refinement has now been implemented within the \ws\ 
code. This will allow to be applied within the fully-featured \ws\ model 
in its present version and as it develops further in the future.

This option is activated by setting the grid string to `{\code QUAD}' in 
{\file ww3\_grid.inp}. A quadtree mesh can be considered as starting from
a standard rectangular grid (at refinement level 0). Any cell of this grid 
may then be refined by subdividing it into four 'child' cells (each at 
refinement level 1) occupying its four corners. 
This process may be repeated as many times as required to reach a desired 
local level of refinement. 
Conversely, this process can be reversed, merging a 'quad' of cells back 
into their common 'parent'. The (locally) fully-refined cells constitute 
the sea points for which the wave action density is computed.

Adaptive mesh refinement means that this process of locally refining and 
coarsening the spatial mesh may be carried out in the course of a simulation. 
Mesh adaptivity is implemented by defining a control variable 
($V(ISEA)$, say) that is computed for each sea point (with 
index $ISEA$), chosen in such a way that finer resolution is required 
where $V$ is large, and lower resolution will suffice where $V$ is small. 
Then upper and lower bounds $V_+$ and $V_-$ are chosen, and the mesh is 
refined wherever $V>V_+$, and coarsened where $V<V_-$.

As an example commonly used in designing static meshes, we might choose 
$V\sqrt(gh) \delta x$, i.e. proportional to a local CFL parameter for 
waves propagating in depth $h$ where the local cell size is $\delta x$.
In that way a uniform level of computational stability is maintained across 
the mesh. 
In that case, of course, the mesh refinement can be completed iteratively 
before we start the simulation. Alternatively, we can choose a variable 
$$V \propto |F''(\delta x)^2| $$
which represents the truncation error in representing a quantity $F$ on a
discrete mesh (the double primes represent the second spatial derivative). 
This would be a suitable choice if $F$ is a quantity we wish to simulate 
as accurately and efficiently as possible. Hence \citep{art:PGRT10xx} 
used significant wave height for $F$ in an adaptivity criterion of this form.
Several choices of adaptivity variable are available in \ws\ . 
Also, either fixed upper and lower bounds may be set by the user, or a 
target total number of sea points may be set, in which case the model 
will adjust the bounds accordingly. 
Note that we should ensure that $V_+/V_- > 4$ to avoid cells
immediately flipping between coarsening and refining.  

As noted above, an adaptive simulation is selected (by setting the grid string 
to `{\code QUAD}' when setting up the model definition by running 
{\code ww3\_grid}. Following that selection, it is necessary to define the 
base (refinement level 0) grid, but for convenience the user instead defines a 
"reference" rectangular grid of size {\code NX} $\times$ {\code NY}, that is 
at some uniform refinement level $p$ (which requires that $2^p$ is a common 
factor of {\code NX} and {\code NY}). Then the bathymetry and, optionally, 
sub-grid obstruction factors on the reference grid may be specified in the 
same way as for a regular grid simulation.

In an adaptive simulation, the spatial structure of the mesh can not be
permanently specified at the model definition stage. 
Rather, it needs to be specified in the various (binary) model output files. 
Hence "gridded" binary output files (grid.ww3) include
representations of the quadtree spatial structure at each time step for
which it has changed. 

As the adaptive grid evolves, it is necessary to provide all relevant input 
fields at the sea points as defined at that point in the computation. That 
includes both static fields (bathymetry, subgrid obstructions), as well
as dynamic inputs (winds, currents, water level, ice). Indeed it is 
anticipated that the adaptive refinement of the wave mesh is likely to 
be strongly controlled by evolving variability the input fields.

On the other hand, the exact spatial mesh on which the wave model will 
need inputs can not be specified in advance.

With these considerations in mind, the binary input formats for input
fields use a similar structure to that of the wave output binaries, 
in which the input fields are represented on a quadtree structure 
derived (by refinement and coarsening) from the common reference grid. 
That is, the wave model retains a distinct quadtree structure for all
required input fields.
Each input quadtree structure is set to local refinement levels matching
the underlying spatial resolution of the corresponding input field.
While noting that a regular grid is a special case of a quadtree 
structure, for which the refinement level is constant,
in the more general case, varying spatial variation of the input fields is 
allowed for.
This anticipates an application where the inputs come from models that
are also adaptive, or are at least in a similar structure. 

So in quadtree simulations, the spatial mesh represented in the model 
definition file applies to the bathymetry data. Normally this would 
represent the most complete representation of the model domain 
than the wave model might use at any given time and place, so that
the ``finest possible'' bathymetry detail is available wherever
it may be needed.

Similarly, binary restart files for adaptive simulations also contain a 
full specification of the spatial mesh at the time of writing. 
Accordingly, it is also now necessary to set the initial spatial
structure of the wave model quadtree mesh as part of the cold start 
process (i.e. in {\code ww3\_strt} rather than in the model definition 
stage (i.e.{\code ww3\_grid}).
This can be achieved by specifying the target number of cells for the
wave model, and the adaptivity criteria to be used to derive the
initial wave quadtree mesh by coarsening the bathymetry quadtree.

A set of pre- and post-processing tools are available to handle input 
and output data on quadtree strucures.

The netcdf preprocessor {\code ww3\_prnc} will convert data from a
sequence of netcdf files, into binary formats readable by \ws\ .
In the input netcdf files, the spatial mesh can be either 
(i) a 2D grid, which can be interpolated onto the quadtree structure 
contained in the model definition file, i.e. the same one used for 
bathymetry data, or
(ii) a one-dimensional structure corresponding to a quadtree structure
obtained from the reference grid. In this no spatial interpolation is done, 
requiring the data to already be on a suitable quadtree mesh.
In both cases, the spatial mesh should remain the same within each 
input netcdf file, but may change from one file to the next.

One anticipated application for the second type is where the inputs 
come from a set of nested 2D grids (in netcdf format). 
With this in mind, a "quadtree preprocessor" 
{\code ww3\_prnq} can be used to spatially interpolate from such a set of 
grids to a quadtree structure. 

Thus, for sets of time-dependent input files, e.g. for wind data,
{\code ww3\_prnq} produces netcdf files at selected time slices with data 
at point locations, and with cell bounds, as given by a fixed quadtree 
structure.

Alternatively, {\code ww3\_prnq} can be used for bathymetry (and, 
optionally, sub-grid obstruction factors).
In this case, output is in an ASCII format, that can be provided as 
an alternative input for {\code ww3\_grid}. The fields are
"pre-averaged" to all lower refinement levels. 

Gridded output from adaptive \ws\ simulations can be handled by 
the netcdf postprocessor {\code ww3\_ounf}, which will 
convert these into a sequence (one per time step) of netcdf files containing 
each field as a one-dimensional array of points, each with specified cell 
centre and bounds.

Two options are available for spatial propagation. A first order scheme 
may be selected by the {\code PR1} switch. If the {\code PR2} switch is used, 
the upstream non-oscillatory 2nd order (UNO) advection scheme \citep{art:Li08} 
is applied, with the same diffusion approach to alleviate the garden 
sprinkler effect as used in the SMC grid, based on the method of 
\cite{art:BH87}simplified to use a homogeneous diffusion parameter $D_{nn}$. 
Treatment of sub-grid obstructions is included for both the first and 
second order schemes. 
This allows the effect of smaller islands, which may be fully 
resolved at higher refinement levels, to still be represented when and 
where the mesh has been coarsened.


