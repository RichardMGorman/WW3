\vssub
\subsubsection{~Quadtree adaptive grids} \label{sub:num_space_qa}
\conthead{\ws\ (NIWA)}{R. Gorman}

\noindent 
Wave simulations are often required to incorporate a wide range of spatial 
scales, in order to capture both long-range swell propagation and finer scale 
interactions either near the coast or within rapidly-varying small-scale 
weather systems.
Computational efficiency can be greatly aided by allowing a model's spatial 
resolution to vary over the domain, with finer resolution where it makes most 
impact on the accuracy of the solution.
In the case where the need for locally higher resolution is static,
for example due to highly variable nearshore topography, this can be addressed 
by nested regular grids, or by various forms of unstructured meshes.
Alternatively, high resolution may be needed to capture the influence of
rapidly varying wind fields in the vicinity of a tropical cyclone.
In this case, we can use adaptive mesh refinement to maintain an efficient 
distribution of spatial resolution.
This has previously been applied to spectral wave modelling 
\citep{art:PGRT10xx} 
by using the Gerris flow solver \citep{art:PopinetXX} to compute the spatial 
advection terms of the action balance action on a quadtree adaptive mesh, coupled to \ws\ V2.22 subroutines to handle a limited set of source terms.

Quadtree adaptive mesh refinement has now been implemented within the \ws\ code. This will allow to be applied within the fully-featured \ws\ model 
in its present version and as it develops further in the future.

This option is activated by setting the grid string to `{\code QUAD}' in 
{\file ww3\_grid.inp}. A quadtree mesh can be considered as starting from
a standard rectangular grid (at refinement level 0). Any cell of this grid may then be refined by subdividing it into four 'child' cells (each at refinement 
level 1) occupying its four corners. This process may be repeated as many 
times as required to reach a desired local level of refinement. Conversely, 
this process can be reversed, merging a 'quad' of cells back into their 
common 'parent'. The (locally) fully-refined cells constitute the sea points 
for which the wave action density is computed.

Adaptive mesh refinement means that ths process of locally refining and 
coarsening the spatial mesh may be carried out in the course of a simulation. 
Mesh adaptivity is implemented by defining a control variable ($V(ISEA)$, say) that is computed for each sea point (with index $ISEA$), chosen in such a 
way that finer resolution is required where $V$ is large, and lower resolution 
will suffice where $V$ is small. Then upper and lower bounds $V_+$ and $V_-$ 
are chosen, and the mesh is refined wherever $V>V_+$, and coarsened where
$V<V_-$>.

As an example commonly used in designing static meshes, we might choose 
$V\sqrt(gh) \delta x$, i.e. proportional to a local CFL parameter for 
waves propagating in depth $h$ where the local cell size is $\delta x$.
In that way a uniform level of computational stability is maintained across 
the mesh. 
In that case, of course, the mesh refinement can be completed iteratively 
before we start the simulation. Alternatively, we can choose a variable 
$$V \propto |F''(\delta x)^2| $$
which represents the truncation error in representing a quantity $F$ on a
discrete mesh (the double primes represent the second spatial derivative). 
This would be a suitable choice if $F$ is a quantity we wish to simulate 
as accurately and efficiently as possible. Hence \citep{art:PGRT10xx} 
used significant wave height for $F$ in an adaptivity criterion of this form.
Several choices of adaptivity variable are available in \ws\. Also, either fixed upper and lower bounds may be set by the user, or a target total number of sea points may be set, in which case the model will adjust the bounds 
accordingly. Note that we should ensure that $V_+/V_- > 4$ to avoid cells
immediately flipping between coarsening and refining.  

As noted above, an adaptive simulation is selected (by setting the grid string 
to `{\code QUAD}' when setting up the model definition by running 
{\code ww3_grid}. Following that selection, it is necessary to define the 
base (refinement level 0) grid, but for convenience the user instead defines a 
"reference" rectangular grid of size {\code NX} $\times$ {\code NY}, that is 
at some uniform refinement level $p$ (which requires that $2^p$ is a common  factor of {\code NX} and {\code NY}). Then the bathymetry and, optionally, 
sub-grid obstruction factors on the reference grid may be specified in the 
same way as for a regular grid simulation.

For adaptive simulations, "gridded" binary output files (grid.ww3) include
representations of the quadtree spatial structure at each time step for
which it has changed. The netcdf postprocessor {\code ww3_ounf} will 
convert these into a sequence (one per time step) of netcdf files containing each field as a one-dimensional array of points, each with specified cell centre and bounds.

As the adaptive grid evolves, it is necessary to provide all relevant input 
fields at the sea points as defined at that point in the computation. That 
includes both static fields (bathymetry, subgrid obstructions), as well
as dynamic inputs (winds, currents, water level, ice). Indeed it is  anticipated that the adaptive refinement of the wave mesh is likely to 
be strongly controlled by evolving variability the input fields.

With this in mind, the binary input formats for input fields uses a similar 
structure to that of the wave output binaries, in which the input fields are 
represented on a quadtree structure derived (by refinement and coarsening) from the common reference grid. This anticipates an application where the inputs 
come from models that are also adaptive, or is at least in a similar
structure. Correspondingly, the netcdf 
preprocessor {\code ww3_prnc} will convert a sequence of netcdf files, 
in which the spatial mesh is either a one-dimensional structure as 
described above for wave outputs, or a 2D grid. The spatial mesh should 
remain the same within each file, but may change from one file to the next.

Another anticipated application is where the inputs come from a set of nested 
2D grids (in netcdf format). With this in mind, a "quadtree preprocessor" 
{\code ww3_prnq} can be used to spatially interpolate from such a set of 
grids to a quadtree structure. For bathymetry (and, optionally, sub-grid
obstruction factors), output is in an ASCII format, that can be provided as 
an alternative input for {\code ww3_grid}. In this case, the fields are
"pre-averaged" to all lower refinement levels. For other, time-dependent, 
sets of input files, the output is in a form suitable as input to 
{\code ww3_prnc}. 

Two options are available for spatial propagation. A first order scheme 
may be selected by the {\code PR1} switch. If the {\code PR2} switch is used,  the upstream non-oscillatory 2nd order (UNO) advection scheme \citep{art:Li08} 
is applied, with the same diffusion approach to alleviate the garden 
sprinkler effect as used in the SMC grid, based on the method of \cite{art:BH87}simplified to use a homogeneous diffusion parameter $D_{nn}$. Treatment 
of sub-grid obstructions is included for both the first and second order 
schemes. This allows the effect of smaller islands, which may be fully 
resolved at higher refinement levels, to still be represented when and 
where the mesh has been coarsened.

restart files

